\documentclass[11pt]{scrartcl}
\usepackage{dominatrix}

\usepackage{solarized-light}
\lstset{
language=R
}

\title{Assignment 3}
\subject{Statistical Finance (STAT W4290)}
\author{Linan Qiu\\\texttt{lq2137}}
\begin{document}
\maketitle

\section*{Ruppert 2}

\subsection*{(a)}

According to CAPM,

\[r = r_f + \alpha(r_m - r_f)\]

If $r_f = 0.03$ and $r_m = 0.14$, for $r$ to be $0.11$, then $\alpha = \frac{8}{11}$. The efficient way to invest is then $\frac{8}{11}$ market portfolio and the rest in risk free assets.

\subsection*{(b)}

Since standard deviation of risk free asset is 0, the standard deviation of the portfolio is $\alpha \sigma_m = \frac{8}{11}0.12 = 0.08727273$

\section*{Ruppert 6}

\subsection*{(a)}

\[\beta = \frac{\sigma_{A,m}}{\sigma^2_m} = \frac{0.0165}{0.11^2} = 1.363636\]

\subsection*{(b)}

\[E(r) = r_f + \beta(r_m - r_f) = 0.04 + 1.363636(0.12-0.04) = 0.1490909\]

\subsection*{(c)}

Proportion of risk due to market risk is $\frac{\sigma_{A,m}}{\sigma_A^2} = \frac{165}{220} = 0.75$

\section*{Ruppert 8}

\subsection*{(a)}

\[E(r_m) = w_C * r_C + w_D * r_D =  0.6*0.04 + 0.4 * 0.06 =0.048\]

\subsection*{(b)}

\[\sigma_{m}^2 = w^2_C\sigma_C + w^2_D\sigma_D + 2w_C w_D \sigma_C \sigma_D \rho_{CD}\]

Then, $\sigma_m = 0.1144727$

\subsection*{(c)}

\[\sigma = \alpha\sigma_m\]

$\alpha = \frac{\sigma}{\sigma_m} = \frac{0.03}{0.1144727} = 0.2620712$. 

Then $1-0.2620712 = 0.7379288$ of equity should be in risk free assets.

\subsection*{(d)}

\[r = r_f + \alpha(r_m - r_f)\]

For $r=0.07$,

\[\alpha = \frac{r-r_f}{r_m - r_f} = \frac{0.07-0.012}{0.048-0.012} = 1.611111\]

This means that $1.611111$ of the equity should be in the risky portfolio, and $-0.611111$ should be in risk free. In the risky portfolio, the proportion of C and D stays the same. Hence, overall:

\begin{itemize}
\item $0.9666666$ in C
\item $0.6444444$ in D
\item $-0.611111$ in risk free assets (short position)
\end{itemize}

\section*{Ruppert 10}

\subsection*{(a)}

\[r = r_f + \alpha(r_m - r_f)\]

For $r = 0.11$,

\[\alpha = \frac{r-r_f}{r_m - r_f} = \frac{0.11 - 0.07}{0.14-0.07} = 0.5714286\]

$0.5714286$ in risky assets, $1-0.5714286 = 0.4285714$ in risk free assets.

\subsection*{(b)}

\[\sigma = \alpha\sigma_m = 0.5714286 * 0.12 = 0.06857143\]

\section*{Additional Problem A1}

\subsection*{(a)}

The general equation for CML is

\[\mu = \mu_f + \frac{\mu_m - \mu_f}{\sigma_m}\sigma\]

The slope is then $\frac{\mu_m - \mu_f}{\sigma_m} = \frac{0.23-0.07}{0.32} = 0.5$

\subsection*{(b)}

\subsubsection*{(i)}

If $\mu=0.39$, and $0.39 = 0.07 + 0.5\sigma$, then $\sigma = 0.64$

\subsubsection*{(ii)}

Honestly I'll use it to pay my tuition. But answering the question, let $\alpha$ be the proportion invested in market portfolio. $\alpha = \frac{\sigma}{\sigma_m} = \frac{0.64}{0.32} = 2$. Then, weight of risk free assets is $-1$. Short \$1000 risk free, long \$2000 risky portfolio.

\subsection*{(c)}

In this case, $\alpha = 0.7$. Given that

\[\mu = \mu_f + \alpha(\mu_m - \mu_f) = 0.182\]

You should expect to have $1000 * (1+0.182) = \$1182$

\section*{Additional Problem A2}

\subsection*{(a)}

\[w_A = \frac{100*1.5}{100*1.5+150*2} = \frac{1}{3}\]

\[w_B = 1 - \frac{1}{3} = \frac{2}{3}\]

\[r_m = \frac{1}{3}0.15 + \frac{2}{3}0.12 = 0.13\]

\subsection*{(b)}

\[\sigma_m^2 = w_A^2 0.15^2 + (1-w_A)^2 0.09^2 + 2w_A(1-w_A)*0.15*0.09*\frac{1}{3} = 0.0081 \]

Then, $\sigma_m = 0.09$.

\subsection*{(c)}

\begin{align*}
\beta_A &= \frac{\sigma_{A,m}}{\sigma_m^2} \\
&= \frac{\frac{1}{3}\sigma_A^2 + \frac{2}{3}\sigma_{A,B}}{\sigma_m^2} \\
&= \frac{\frac{1}{3}\sigma_A^2 + \frac{2}{3}\sigma_A\sigma_B\rho_{AB}}{\sigma_m^2} \\
&= 1.296296
\end{align*}

\subsection*{(d)}

\[r_A = r_f + \beta_A (r_m - r_f)\]

It's late and I can't do math anymore, so...

\begin{lstlisting}
> f = function(x) {x + 1.296296*(0.13-x) - 0.15}
> uniroot(f, c(-1000, 1000))
$root
[1] 0.06249993

$f.root
[1] -2.939315e-14

$iter
[1] 2

$init.it
[1] NA

$estim.prec
[1] 6.103516e-05
\end{lstlisting}

\[r_f = 0.06249993\]

\hrulefill

I initially thought we had to calculate the tangency portfolio on our own. If this is the case, please use this answer instead:

To calculate the tangency portfolio, let $w_A$ and $w_B$ be weights of A and B in the tangency portfolio. $w_B = 1-w_A$.

\[r_m = w_A 0.15 + (1-w_A) 0.12\]

and

\[\sigma_m^2 = w_A^2 0.15^2 + (1-w_A)^2 0.09^2 + 2w_A(1-w_A)*0.15*0.09*\frac{1}{3}\]

$w_A$ of the tangency portfolio is the one that minimizes $\sigma_m^2$.

\begin{lstlisting}
> sigmam2 = function(w) {w^2 * 0.15^2 + (1-w)^2 * 0.09^2 + 2*w*(1-w)*0.15*0.09/3}
> optimize(sigmam2, interval=c(-100,100), maximum=FALSE)
$minimum
[1] 0.1666667

$objective
[1] 0.0075
\end{lstlisting}

$w_A = 0.1666667$ with an accompanying $\sigma_m^2 = 0.0075$
\[r_m = w_A 0.15 + (1-w_A)0.12 = 0.125\]

As calculated earlier, $\sigma_m^2 = 0.0075$. Hence, 

\[\sigma_m = \sqrt{ w_A^2 0.15^2 + (1-w_A)^2 0.09^2 + 2w_A(1-w_A)*0.15*0.09*\frac{1}{3}} = 0.08660254\]


\end{document}