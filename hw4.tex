\documentclass[11pt]{scrartcl}
\usepackage{dominatrix}

\newcommand{\PAR}{\ensuremath{\mathrm{PAR}}}

\usepackage{solarized-light}
\lstset{
language=R
}

\title{Assignment 4}
\subject{Statistical Finance (STAT W4290)}
\author{Linan Qiu\\\texttt{lq2137}}
\begin{document}
\maketitle

\section*{Rupert Problem 3}

\[1200 = \frac{40}{r}\left(1 - \frac{1}{(1+r)^{30}} \right) + \frac{1000}{(1+r)^{30}}\]

Solving for $r$, we get $r=0.03241618$

\begin{lstlisting}
> zero = function(r) {(40/r)*(1-1/(1+r)^60) + 1000/(1+r)^60 - 1200}
> uniroot(zero, c(0.03, 0.04), maxiter=100)
$root
[1] 0.03241618

$f.root
[1] -0.5497406

$iter
[1] 3

$init.it
[1] NA

$estim.prec
[1] 6.103516e-05
\end{lstlisting}

\section*{Rupert Problem 4}

\[9800 = \frac{280}{\frac{r}{2}} \left(1-\frac{1}{\left(1+\frac{r}{2}\right)^{8*2}}\right) + \frac{10000}{\left(1+\frac{r}{2}\right)^{8*2}}\]

Solving for $r$, we get

\begin{lstlisting}
> zero = function(r) {(280/(r/2)) * (1-1/(1+(r/2))^(8*2)) + 10000/(1+(r/2))^(8*2) - 9800}
> uniroot(zero, c(0.05, 0.06), maxiter=100)
$root
[1] 0.05920323

$f.root
[1] -1.794659

$iter
[1] 2

$init.it
[1] NA

$estim.prec
[1] 6.103516e-05
\end{lstlisting}

\section*{Rupert Exercise 1}

\[y(t) = \frac{1}{t} \int_0^t r(s) \: ds\]

\subsection*{(a)}

Setting $t=20$,

\begin{align*}
y(20) &= \frac{1}{20} \int_0^t 0.028 + 0.00042s \: ds \\
&=\frac{1}{20}(0.028*20 + 0.00021*(20^2)) \\
&=0.0322
\end{align*}

\subsection*{(b)}

\[y(15) = 0.03115\]

\begin{lstlisting}
> r = function(t) {0.028 + 0.00042*t}
> y = integrate(r, 0, 15)$value / 15
> y
[1] 0.03115
\end{lstlisting}

Then the price of the bond is 

\[P = \frac{1000}{(1+0.03115)^{15}} = 631.2077\]

\section*{Rupert Exercise 3}

\[y(t) = \frac{1}{t} \int_0^t r(s) \: ds\]

\subsection*{(a)}

Setting $t=5$, $y(5) = 0.03616667$

\begin{lstlisting}
> r = function(t){0.032+0.001*t+0.0002*t^2}
> y = integrate(r, 0, 5)$value / 5
> y
[1] 0.03616667
\end{lstlisting}

\subsection*{(b)}

Assume bond has par \$100, then

\[P = \frac{100}{(1+0.03616667)^5} = 83.72438\]

\section*{Ruppert Exercise 5}

Now the sum of a geometric series $C + Cm + Cm^2 + ... = \frac{C}{1-m}$. If $m = \frac{1}{1-r}$, then $C + Cm + Cm^2 + ... = \frac{C(1+r)}{r}$. If the cash flow $C$ begins a year later, we discount the entire amount by $(1+r)$ thereby making the present value of the sum $\frac{C}{r}$.

Then,

\begin{align*}
&\sum_{t=1}^{2T} \frac{C}{(1+r)^t} + \frac{\PAR}{(1+r)^{2T}} \\
&=\sum_{t=1}^\infty \frac{C}{(1+r)^t} - \sum_{t=2T+1}^\infty \frac{C}{(1+r)^t} + \frac{\PAR}{(1+r)^{2T}} \\
&= \frac{C}{r} - \frac{C}{r(1+r)^{2T}} + \frac{\PAR}{(1+r)^{2T}} \\
&= \frac{C}{r} + \left(\PAR - \frac{C}{r}\right)(1+r)^{-2T}
\end{align*}

\section*{Ruppert Exercise 7}

\subsection*{(a)}

Given that

\begin{align*}
818 &= \frac{1000}{\exp{5r}} \\
r &= \frac{\log{\frac{1000}{818}}}{5}
&= 0.04017859
\end{align*}

This means that $r = 0.04017859$

\subsection*{(b)}

\[P = \frac{1000}{\exp{4*0.042}} = 845.3538\]

\subsection*{(c)}

\[R_2 = \frac{P_2}{P_1} - 1 = \frac{845.3538}{818} - 1 = 0.03343985\]

\section*{Ruppert Exercise 8}

\subsection*{(a)}

\[P = \frac{22}{\frac{0.04}{2}}\left(1-\frac{1}{1.02^{20}}\right) + \frac{1000}{1.02^{20}} = 1032.703\]

\subsection*{(b)}

Coupon rate is $\frac{22}{1000} = 0.022$. Coupon rate is higher than yield rate of $0.02$, hence selling above par.

\section*{Ruppert Exercise 9}

\subsection*{(a)}

\[1050 = \frac{24}{\frac{r}{2}}\left(1-\frac{1}{\left(1+\frac{r}{2}\right)^{14}}\right) + \frac{1000}{\left(1+\frac{r}{2}\right)^{14}}\]

Solving for $r$, we get $r=0.03975274$

\begin{lstlisting}
> zero = function(r) {(24/(r/2))*(1-1/(1+r/2)^14) + 1000/(1+r/2)^14 - 1050}
> uniroot(zero, c(0.001, 1))
$root
[1] 0.03975274

$f.root
[1] -0.03484846

$iter
[1] 5

$init.it
[1] NA

$estim.prec
[1] 6.103516e-05
\end{lstlisting}

\subsection*{(b)}

Current price is \$1050. Coupon payment is \$24. Current yield is $2*\frac{24}{1050} = 0.04571429$

\subsection*{(c)}

Yield to maturity is less than current yield. This is because the bond is selling at above par (price $>$ face value), hence the eventual face value principal payment would be discounted at a level steeper than the current yield, thereby resulting in a lower yield to maturity.

\section*{Ruppert Exercise 14}

\subsection*{(a)}

\[Y_T = 0.04 + 0.001T\]

Then,

\[P = \frac{1000}{(1+Y_T(10))^{10}} = 613.9133\]

\subsection*{(b)}

\[P = \frac{1000}{(1+Y_T(9))^9} = 639.1099\]

Then, return $R_2$ is

\[R_2 = \frac{P_2}{P_1} - 1 = \frac{639.1099}{613.9133} - 1 = 0.0410426\]

\section*{Ruppert Exercise 16}

\subsection*{(a)}

To find the 5 year yield to maturity,

\[y(t) = \int_0^t r(s) \: ds = \int_0^t 0.03 + 0.001s + 0.0002s^2 \: ds\]

Then, $y(5) = 0.03416667$

\begin{lstlisting}
> r = function(s) {0.03 + 0.001*s + 0.0002*s^2}
> integrate(r, 0, 5)$value/5
[1] 0.03416667
\end{lstlisting}

\subsection*{(b)}

Assume face value of \$100,

\[P = \frac{100}{(1+0.03416667)^5} = 84.53709\]

\section*{Ruppert Exercise 20}

\subsection*{(a)}

To find the 4 year yield to maturity,

\[y(t) = \int_0^t r(s) \: ds = \int_0^t 0.022 + 0.005s - 0.004s^2 + 0.0003s^3 \: ds\]

Then,

\[P = \sum_{t=1}^8 \frac{21}{\left(1+y\left(\frac{t}{2}\right)\right)^t} + \frac{1000}{\left(1+y\left(\frac{8}{2}\right)\right)^8}\]

\begin{lstlisting}
> r = function(s) {0.022 + 0.005*s - 0.004*s^2 + 0.0003*s^3}
> y = function(t) {integrate(r, 0, t)$value/t}
> yields = sapply(n/2, y)
> yields
[1] 0.02292604 0.02324167 0.02300312 0.02226667 0.02108854 0.01952500 0.01763229 0.01546667
> pv_coupon = mapply(function(x, yield) {21/(1+yield)^x}, n, yields, SIMPLIFY=TRUE)
> pv_coupon
[1] 20.52934 20.05686 19.61500 19.22926 18.91918 18.69959 18.58160 18.57351
> pv_face = 1000/(1+y(4))^8
> sum(pv_coupon) + pv_face
[1] 1038.657
\end{lstlisting}

The price of the bond is \$1038.657.

\subsection*{(b)}

Assume that we are asking for \textbf{Macaulay duration} (not modified duration)

\[D = \frac{\sum_{t=1}^8 \left[\frac{t}{2} \left(\frac{21}{\left(1+y\left(\frac{t}{2}\right)\right)^t} \right)\right] + 4 * \frac{1000}{(1+y(4))^8}}{P}\]

\begin{lstlisting}
> coupons = sapply(n, function(x) {(x/2) * (21/(1+y(x/2))^x)})
> coupons
[1] 10.26467 20.05686 29.42251 38.45851 47.29795 56.09876 65.03560 74.29403
> sum(coupons) + 4*pv_face
[1] 3878.74
> (sum(coupons) + 4*pv_face)/(sum(pv_coupon) + pv_face)
[1] 3.73438
\end{lstlisting}

The duration is $3.73438$

\end{document}